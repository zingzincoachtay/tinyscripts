\documentclass{article}
\usepackage[dvips]{graphicx}
\begin{document}

\section*{Guitar}
The scale of guitar is made by pressing frets to change the length of string. Assume, we make "do" without pressing any frets. We make a sound one octave higher by pressing at the midpoint.

%\begin{figure}[htbp]
%\begin{center}
%\includegraphics{guitar.jpg}
%\end{center}
%\end{figure}

There are 12 frets in BC. The ratio of length from A to one of frets and to right of the fret ($\frac{yellow dot on left}{yellow dot on right}$) is approximately .94. It's true for any combination of frets next to each other.

Therefore, in order to obtain a value of length between A and the fret on the left, you multiply a certain ratio (r) with a length between A and the fret on the right. The question is what would be the ratio (r). It's easy. Let's say ... L is the length of the whole string. The length between A and the first fret is L$\times$r, the length between A and the second fret is L$\times r^2$, since new length is a ratio (r) $\times$ last length. You continue same things until you finish about all 12 frets and reach the midpoint. Since the midpoint is expressed as, $\frac{1}{2}a$, then:
\[a\times r\times r\times r\times r\times r\times r\times r\times r\times r\times r\times r\times r\times r\times = \frac{1}{2}a\]
\[r^{12} = \frac{1}{2}\]
\[r = \sqrt[12]{\frac{1}{2}} = .9438743127\]

The scale created by this is called the equal temperament scale. In about 16th century, this is invented. This scale is still used for the scale of instruments, even now. 

\pagebreak

\section*{Cembalo and Piano}
If you play piano, you know the instrument called Cembalo. A cembalo is the ancestor of a piano. The lengths of the cords of 'do' in each octave are doubled. 
\begin{table}
\begin{center}
\begin{tabular}{|c|c|c|c|c|c|}
Length & L & 2L & $2^2L$ & $2^3L$ & $2^4L$\\
Log & $log L$ & $log 2L$ & $log 2^2L$ & $log 2^3L$ & $log 2^4L$\\
\end{tabular}
\end{center}
\end{table}

As you can see, the length of the cembalo, (or rather the shape of the cembalo), is an exponential function. Furthermore, the keyboard is a logarithmic function. 
\pagebreak

\section*{Pythagorian Scale}


\end{document}
